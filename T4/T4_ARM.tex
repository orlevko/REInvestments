\documentclass[ignorenonframetext,]{beamer}
\setbeamertemplate{caption}[numbered]
\setbeamertemplate{caption label separator}{: }
\setbeamercolor{caption name}{fg=normal text.fg}
\beamertemplatenavigationsymbolsempty
\usepackage{lmodern}
\usepackage{amssymb,amsmath}
\usepackage{ifxetex,ifluatex}
\usepackage{fixltx2e} % provides \textsubscript
\ifnum 0\ifxetex 1\fi\ifluatex 1\fi=0 % if pdftex
\usepackage[T1]{fontenc}
\usepackage[utf8]{inputenc}
\else % if luatex or xelatex
\ifxetex
\usepackage{mathspec}
\else
\usepackage{fontspec}
\fi
\defaultfontfeatures{Ligatures=TeX,Scale=MatchLowercase}
\fi
% use upquote if available, for straight quotes in verbatim environments
\IfFileExists{upquote.sty}{\usepackage{upquote}}{}
% use microtype if available
\IfFileExists{microtype.sty}{%
\usepackage{microtype}
\UseMicrotypeSet[protrusion]{basicmath} % disable protrusion for tt fonts
}{}
\newif\ifbibliography
\usepackage{color}
\usepackage{fancyvrb}
\newcommand{\VerbBar}{|}
\newcommand{\VERB}{\Verb[commandchars=\\\{\}]}
\DefineVerbatimEnvironment{Highlighting}{Verbatim}{commandchars=\\\{\}}
% Add ',fontsize=\small' for more characters per line
\usepackage{framed}
\definecolor{shadecolor}{RGB}{248,248,248}
\newenvironment{Shaded}{\begin{snugshade}}{\end{snugshade}}
\newcommand{\KeywordTok}[1]{\textcolor[rgb]{0.13,0.29,0.53}{\textbf{{#1}}}}
\newcommand{\DataTypeTok}[1]{\textcolor[rgb]{0.13,0.29,0.53}{{#1}}}
\newcommand{\DecValTok}[1]{\textcolor[rgb]{0.00,0.00,0.81}{{#1}}}
\newcommand{\BaseNTok}[1]{\textcolor[rgb]{0.00,0.00,0.81}{{#1}}}
\newcommand{\FloatTok}[1]{\textcolor[rgb]{0.00,0.00,0.81}{{#1}}}
\newcommand{\ConstantTok}[1]{\textcolor[rgb]{0.00,0.00,0.00}{{#1}}}
\newcommand{\CharTok}[1]{\textcolor[rgb]{0.31,0.60,0.02}{{#1}}}
\newcommand{\SpecialCharTok}[1]{\textcolor[rgb]{0.00,0.00,0.00}{{#1}}}
\newcommand{\StringTok}[1]{\textcolor[rgb]{0.31,0.60,0.02}{{#1}}}
\newcommand{\VerbatimStringTok}[1]{\textcolor[rgb]{0.31,0.60,0.02}{{#1}}}
\newcommand{\SpecialStringTok}[1]{\textcolor[rgb]{0.31,0.60,0.02}{{#1}}}
\newcommand{\ImportTok}[1]{{#1}}
\newcommand{\CommentTok}[1]{\textcolor[rgb]{0.56,0.35,0.01}{\textit{{#1}}}}
\newcommand{\DocumentationTok}[1]{\textcolor[rgb]{0.56,0.35,0.01}{\textbf{\textit{{#1}}}}}
\newcommand{\AnnotationTok}[1]{\textcolor[rgb]{0.56,0.35,0.01}{\textbf{\textit{{#1}}}}}
\newcommand{\CommentVarTok}[1]{\textcolor[rgb]{0.56,0.35,0.01}{\textbf{\textit{{#1}}}}}
\newcommand{\OtherTok}[1]{\textcolor[rgb]{0.56,0.35,0.01}{{#1}}}
\newcommand{\FunctionTok}[1]{\textcolor[rgb]{0.00,0.00,0.00}{{#1}}}
\newcommand{\VariableTok}[1]{\textcolor[rgb]{0.00,0.00,0.00}{{#1}}}
\newcommand{\ControlFlowTok}[1]{\textcolor[rgb]{0.13,0.29,0.53}{\textbf{{#1}}}}
\newcommand{\OperatorTok}[1]{\textcolor[rgb]{0.81,0.36,0.00}{\textbf{{#1}}}}
\newcommand{\BuiltInTok}[1]{{#1}}
\newcommand{\ExtensionTok}[1]{{#1}}
\newcommand{\PreprocessorTok}[1]{\textcolor[rgb]{0.56,0.35,0.01}{\textit{{#1}}}}
\newcommand{\AttributeTok}[1]{\textcolor[rgb]{0.77,0.63,0.00}{{#1}}}
\newcommand{\RegionMarkerTok}[1]{{#1}}
\newcommand{\InformationTok}[1]{\textcolor[rgb]{0.56,0.35,0.01}{\textbf{\textit{{#1}}}}}
\newcommand{\WarningTok}[1]{\textcolor[rgb]{0.56,0.35,0.01}{\textbf{\textit{{#1}}}}}
\newcommand{\AlertTok}[1]{\textcolor[rgb]{0.94,0.16,0.16}{{#1}}}
\newcommand{\ErrorTok}[1]{\textcolor[rgb]{0.64,0.00,0.00}{\textbf{{#1}}}}
\newcommand{\NormalTok}[1]{{#1}}

% Prevent slide breaks in the middle of a paragraph:
\widowpenalties 1 10000
\raggedbottom

\AtBeginPart{
\let\insertpartnumber\relax
\let\partname\relax
\frame{\partpage}
}
\AtBeginSection{
\ifbibliography
\else
\let\insertsectionnumber\relax
\let\sectionname\relax
\frame{\sectionpage}
\fi
}
\AtBeginSubsection{
\let\insertsubsectionnumber\relax
\let\subsectionname\relax
\frame{\subsectionpage}
}

\setlength{\parindent}{0pt}
\setlength{\parskip}{6pt plus 2pt minus 1pt}
\setlength{\emergencystretch}{3em}  % prevent overfull lines
\providecommand{\tightlist}{%
\setlength{\itemsep}{0pt}\setlength{\parskip}{0pt}}
\setcounter{secnumdepth}{0}

\title{Adjustable rate mortgages}
\subtitle{Real-Estate Investments - Tutorial 4}
\author{Or Levkovich}
\date{29 September 2017}

\begin{document}
\frame{\titlepage}

\begin{frame}{Adjustable rate mortgages (ARM)}

Recall, mortgage pricing follows : \(i = r + f + p\)

\begin{itemize}
\tightlist
\item
  Real rate of interest (\(r\)).
\item
  Inflation (\(f\))
\item
  Risk premium (\(p\))
\end{itemize}

\end{frame}

\begin{frame}{Adjustable rate mortgages (ARM)}

\(i = r + f + p\)

\begin{itemize}
\item
  \(f\) is adjustable: \textbf{Price level adjusted mortgage (PLAM)}
\item
  Problems:

  \begin{itemize}
  \tightlist
  \item
    Payment is computed again every period given changes in CPI.
  \item
    CPI may not be appropriate to determine mortgage pricing.
  \item
    Borrower's income might not follow the rate inflation.
  \item
    Past changes in CPI affect future mortgage interest rate.
  \end{itemize}
\end{itemize}

\end{frame}

\begin{frame}{Adjustable rate mortgages (ARM)}

\(i = r + f + p\)

\begin{itemize}
\tightlist
\item
  \(r, f, p\) are adjustable: rate is linked to \textbf{index interest
  rate}.

  \begin{itemize}
  \tightlist
  \item
    Let market expectations of \(r, p, f\) determine \(i\).
  \item
    Mortgage interest rates reflects future expectations and not past
    outcomes.
  \end{itemize}
\end{itemize}

Important observations:

\begin{itemize}
\tightlist
\item
  Compared with FRM, lender faces less risk and borrower bears more
  risk.
\item
  Risk depends on the adjustment intervals.
\end{itemize}

\end{frame}

\begin{frame}{Adjustable rate mortgages (ARM)}

\(i = r + f + p\)

\begin{itemize}
\tightlist
\item
  \textbf{Hybrid ARM}

  \begin{itemize}
  \tightlist
  \item
    ARM rates adjust in relatively long intervals (i.e, every few
    years).
  \item
    Interest rate risk is better divided between borrower and lender
    (compared with other ARM).
  \end{itemize}
\item
  Risk to lender:
\end{itemize}

\small

\emph{FRM} \(>\) \emph{ARM (Hybrid - capped interest rates)} \(>\)
\emph{FRM (Hybrid - capped payments)} \(>\) \emph{ARM (no limit)}

\normalsize

(Lower risk = lower demanded premium)

\end{frame}

\begin{frame}{ARM main characteristics and terms}

\begin{itemize}
\tightlist
\item
  \textbf{Index}: The index to which the interest rate is linked.
\item
  \textbf{Margin}: A premium in addition to the index chosen.
\item
  \textbf{Cap/Floor}: Limitations on the changes in payments or interest
  rates.

  \begin{itemize}
  \tightlist
  \item
    Due to caps, payments may be lower than the interest on the
    outstanding loan \(\rightarrow\) negative amortization.
  \end{itemize}
\item
  \textbf{Reset}: The point in time when mortgage payments will be
  adjusted.
\item
  \textbf{Discount points}: Points or fees added to increase the
  lender's yield (also in FRM).
\end{itemize}

\end{frame}

\section{Exercises (from the book)}\label{exercises-from-the-book}

\begin{frame}{Problem 2 (P.146)}

\small

A basic ARM is made for \$200,000 at an initial interest rate of 6
percent for 30 years with an annual reset date. The borrower believes
that the interest rate at the beginning of year 2 will increase to 7
percent.

\begin{enumerate}
\def\labelenumi{\arabic{enumi}.}
\tightlist
\item
  Assuming that a fully amortizing loan is made, what will monthly
  payments be during year 1?
\item
  Based on (1) what will the loan balance be at the end of year 1?
\item
  Given that the interest rate is expected to be 7 percent at the
  beginning of year 2, what will monthly payments be during year 2?
\item
  What will be the loan balance at the end of year 2?
\item
  What would be the monthly payments in year 1 if they are to be
  interest only?
\item
  Assuming terms in (5), what would monthly interest only payments be in
  year 2?
\end{enumerate}

\end{frame}

\begin{frame}[fragile]{Problem 2 answer}

\begin{Shaded}
\begin{Highlighting}[]
\NormalTok{See Excel}

\CommentTok{# 1}
\NormalTok{=}\KeywordTok{PMT}\NormalTok{(}\FloatTok{0.06}\NormalTok{/}\DecValTok{12}\NormalTok{, }\DecValTok{30}\NormalTok{*}\DecValTok{12}\NormalTok{, }\DecValTok{200000}\NormalTok{, }\DecValTok{0}\NormalTok{) =}\StringTok{ }\NormalTok{-}\DecValTok{1}\NormalTok{,}\FloatTok{199.1}

\CommentTok{# 2}
\NormalTok{=}\KeywordTok{FV}\NormalTok{(}\FloatTok{0.06}\NormalTok{/}\DecValTok{12}\NormalTok{, }\DecValTok{12}\NormalTok{, -}\FloatTok{1199.1}\NormalTok{, }\DecValTok{200000}\NormalTok{) =}\StringTok{ }\NormalTok{-}\DecValTok{197}\NormalTok{,}\DecValTok{543}

\CommentTok{# 3}
\NormalTok{=}\KeywordTok{PMT}\NormalTok{(}\FloatTok{0.07}\NormalTok{/}\DecValTok{12}\NormalTok{, }\DecValTok{29}\NormalTok{*}\DecValTok{12}\NormalTok{, }\DecValTok{197543}\NormalTok{, }\DecValTok{0}\NormalTok{) =}\StringTok{ }\NormalTok{-}\DecValTok{1}\NormalTok{,}\FloatTok{327.7}
\end{Highlighting}
\end{Shaded}

\end{frame}

\begin{frame}[fragile]{Problem 2 answer (continued)}

\begin{Shaded}
\begin{Highlighting}[]
\CommentTok{# 4}
\NormalTok{=}\KeywordTok{FV}\NormalTok{(}\FloatTok{0.07}\NormalTok{/}\DecValTok{12}\NormalTok{, }\DecValTok{12}\NormalTok{, -}\FloatTok{1327.7}\NormalTok{, }\DecValTok{197543}\NormalTok{) =}\StringTok{ }\NormalTok{-}\DecValTok{195}\NormalTok{,}\DecValTok{370}

\CommentTok{# 5 }
\NormalTok{=}\KeywordTok{PMT}\NormalTok{(}\FloatTok{0.06}\NormalTok{/}\DecValTok{12}\NormalTok{, }\DecValTok{30}\NormalTok{*}\DecValTok{12}\NormalTok{, }\DecValTok{200000}\NormalTok{, -}\DecValTok{200000}\NormalTok{) =}\StringTok{ }\NormalTok{-}\DecValTok{1}\NormalTok{,}\DecValTok{000}

\CommentTok{# 6}
\NormalTok{=}\KeywordTok{PMT}\NormalTok{(}\FloatTok{0.07}\NormalTok{/}\DecValTok{12}\NormalTok{, }\DecValTok{29}\NormalTok{*}\DecValTok{12}\NormalTok{, }\DecValTok{200000}\NormalTok{, -}\DecValTok{200000}\NormalTok{) =}\StringTok{ }\NormalTok{-}\DecValTok{1}\NormalTok{,}\DecValTok{167}
\end{Highlighting}
\end{Shaded}

\normalsize

\end{frame}

\begin{frame}{Problem 3 (P.147)}

\small

A 3/1 ARM (reset after 3 years) is made for \$150,000 at 7\% with a
30-year maturity.

\begin{enumerate}
\def\labelenumi{\alph{enumi}.}
\tightlist
\item
  Assuming that fixed payments are to be made monthly for three years
  and that the loan is fully amortizing, what will be the monthly
  payments? What will be the loan balance after three years?
\item
  What would new payments be beginning in year 4 if the interest rate
  fell to 6\% and the loan continued to be fully amortizing?
\item
  In (1) what would monthly payments be during year 1 if they were
  interest only? What would payments be beginning in year 4 if interest
  rates fell to 6\% and the loan became fully amortizing?
\end{enumerate}

\end{frame}

\begin{frame}[fragile]{Problem 3 answer}

\begin{Shaded}
\begin{Highlighting}[]
\NormalTok{See Excel}

\CommentTok{# 1}
\NormalTok{=}\KeywordTok{PMT}\NormalTok{(}\FloatTok{0.07}\NormalTok{/}\DecValTok{12}\NormalTok{, }\DecValTok{30}\NormalTok{*}\DecValTok{12}\NormalTok{, }\DecValTok{150000}\NormalTok{, }\DecValTok{0}\NormalTok{) =}\StringTok{ }\NormalTok{-}\FloatTok{997.9537}
\NormalTok{=}\KeywordTok{FV}\NormalTok{(}\FloatTok{0.07}\NormalTok{/}\DecValTok{12}\NormalTok{, }\DecValTok{12}\NormalTok{*}\DecValTok{3}\NormalTok{, -}\FloatTok{997.9537}\NormalTok{, }\DecValTok{150000}\NormalTok{) =}\StringTok{ }\NormalTok{-}\DecValTok{145}\NormalTok{,}\FloatTok{090.4}

\CommentTok{# 2}
\NormalTok{=}\KeywordTok{PMT}\NormalTok{(}\FloatTok{0.06}\NormalTok{/}\DecValTok{12}\NormalTok{, }\DecValTok{27}\NormalTok{*}\DecValTok{12}\NormalTok{, }\FloatTok{145090.4}\NormalTok{, }\DecValTok{0}\NormalTok{) =}\StringTok{ }\NormalTok{-}\FloatTok{905.34}

\CommentTok{# 3}
\NormalTok{=}\KeywordTok{PMT}\NormalTok{(}\FloatTok{0.07}\NormalTok{/}\DecValTok{12}\NormalTok{, }\DecValTok{30}\NormalTok{*}\DecValTok{12}\NormalTok{, }\DecValTok{150000}\NormalTok{, -}\DecValTok{150000}\NormalTok{) =}\StringTok{ }\NormalTok{-}\DecValTok{875}
\NormalTok{=}\KeywordTok{FV}\NormalTok{(}\FloatTok{0.07}\NormalTok{/}\DecValTok{12}\NormalTok{, }\DecValTok{12}\NormalTok{*}\DecValTok{3}\NormalTok{, -}\DecValTok{875}\NormalTok{, }\DecValTok{150000}\NormalTok{) =}\StringTok{ }\NormalTok{-}\DecValTok{150}\NormalTok{,}\DecValTok{000}
\NormalTok{=}\KeywordTok{PMT}\NormalTok{(}\FloatTok{0.06}\NormalTok{/}\DecValTok{12}\NormalTok{, }\DecValTok{27}\NormalTok{*}\DecValTok{12}\NormalTok{, }\DecValTok{150000}\NormalTok{, }\DecValTok{0}\NormalTok{) =}\StringTok{ }\NormalTok{-}\FloatTok{935.978}
\end{Highlighting}
\end{Shaded}

\normalsize

\end{frame}

\begin{frame}{Problem 5 (P.147)}

\small

An interest only ARM is made for \$200,000 for 30 years. The start rate
is 5\% and the borrower will make monthly interest only payments for 3
years. Payments thereafter must be sufficient to fully amortize the loan
at maturity.

\begin{enumerate}
\def\labelenumi{\alph{enumi}.}
\tightlist
\item
  If the borrower makes interest only payments for 3 years, what will
  payments be?
\item
  Assume that at the end of year 3, the reset rate is 6\%. The borrower
  must now make payments so as to fully amortize the loan. What will
  payments be?
\end{enumerate}

\end{frame}

\begin{frame}[fragile]{Problem 5 answer}

\begin{Shaded}
\begin{Highlighting}[]
\CommentTok{# 1}
\NormalTok{=}\KeywordTok{PMT}\NormalTok{(}\FloatTok{0.05}\NormalTok{/}\DecValTok{12}\NormalTok{, }\DecValTok{30}\NormalTok{*}\DecValTok{12}\NormalTok{, }\DecValTok{200000}\NormalTok{, -}\DecValTok{200000}\NormalTok{) =}\StringTok{ }\NormalTok{-}\FloatTok{833.3333}

\CommentTok{# 2}
\NormalTok{=}\KeywordTok{PMT}\NormalTok{(}\FloatTok{0.06}\NormalTok{/}\DecValTok{12}\NormalTok{, }\DecValTok{27}\NormalTok{*}\DecValTok{12}\NormalTok{, }\DecValTok{200000}\NormalTok{, }\DecValTok{0}\NormalTok{) =}\StringTok{ }\NormalTok{-}\DecValTok{1}\NormalTok{,}\FloatTok{247.971}
\end{Highlighting}
\end{Shaded}

\normalsize

\end{frame}

\begin{frame}{Problem 6 (P.147)}

\footnotesize

A borrower has been analyzing different ARM alternatives for the
purchase of a property. The borrower anticipates owning the property for
five years. The lender first offers a \$150,000, 30-year fully
amortizing ARM with the following terms:

\begin{itemize}
\tightlist
\item
  Initial interest rate = 6\%
\item
  Payments = Reset each year
\item
  Margin = 2\%
\item
  Negative amortization = Not allowed
\item
  Discount points = 2\%
\end{itemize}

Based on estimated forward rates, the index to which the ARM is tied is
forecasted as follows:

\begin{itemize}
\tightlist
\item
  Beginning of year 2 = 7\%
\item
  Beginning of year 3 = 8.5\%
\item
  Beginning of year 4 = 9.5\%
\item
  Beginning of year 5 = 11\%
\end{itemize}

Compute the payments, loan balances, and yield for the unrestricted ARM
for the five-year period.

\end{frame}

\begin{frame}[fragile]{Problem 6 answer}

\begin{Shaded}
\begin{Highlighting}[]
\NormalTok{See excel.}
\end{Highlighting}
\end{Shaded}

\normalsize

\end{frame}

\begin{frame}{Problem 10 (P.147)}

\small

A floating rate mortgage loan is made for \$100,000 for a 30-year period
at an initial rate of 12\% interest. However, the borrower and lender
have negotiated a monthly payment of \$800.

\begin{enumerate}
\def\labelenumi{\alph{enumi}.}
\tightlist
\item
  What will be the loan balance at the end of year 1?
\item
  What if the interest rate increases to 13\% at the end of year 1? How
  much interest will be accrued as negative amortization in year 1 if
  the payment remains at \$800? Year 5?
\end{enumerate}

\end{frame}

\begin{frame}[fragile]{Problem 10 answer}

\begin{Shaded}
\begin{Highlighting}[]
\CommentTok{# 1}
\NormalTok{=}\KeywordTok{FV}\NormalTok{(}\FloatTok{0.12}\NormalTok{/}\DecValTok{12}\NormalTok{, }\DecValTok{12}\NormalTok{*}\DecValTok{1}\NormalTok{, -}\DecValTok{800}\NormalTok{, }\DecValTok{100000}\NormalTok{) =}\StringTok{ }\NormalTok{-}\DecValTok{102}\NormalTok{,}\FloatTok{536.5} \NormalTok{(}\CommentTok{#year 1)}

\CommentTok{#2}
\NormalTok{=}\KeywordTok{FV}\NormalTok{(}\FloatTok{0.13}\NormalTok{/}\DecValTok{12}\NormalTok{, }\DecValTok{12}\NormalTok{*}\DecValTok{1}\NormalTok{, -}\DecValTok{800}\NormalTok{, }\FloatTok{102536.5}\NormalTok{) =}\StringTok{ }\NormalTok{-}\DecValTok{106}\NormalTok{,}\FloatTok{496.7} \NormalTok{(}\CommentTok{# year 2)}
\NormalTok{=}\KeywordTok{FV}\NormalTok{(}\FloatTok{0.13}\NormalTok{/}\DecValTok{12}\NormalTok{, }\DecValTok{12}\NormalTok{*}\DecValTok{5}\NormalTok{, -}\DecValTok{800}\NormalTok{, }\FloatTok{102536.5}\NormalTok{) =}\StringTok{ }\NormalTok{-}\DecValTok{128}\NormalTok{,}\FloatTok{611.9} \NormalTok{(}\CommentTok{# year 6)}
\end{Highlighting}
\end{Shaded}

\normalsize

\end{frame}

\begin{frame}{Problem 1 (P.177)}

\small
A borrower can obtain an 80\% loan with an 8\% interest rate and monthly
payments. The loan is to be fully amortized over 25 years.
Alternatively, he could obtain a 90\% loan at an 8.5\% rate with the
same loan term. The borrower plans to own the property for the entire
loan term.

\begin{enumerate}
\def\labelenumi{\arabic{enumi}.}
\tightlist
\item
  What is the incremental cost of borrowing the additional funds? (Hint:
  The dollar amount of the loan doesn't affect the answer.)
\item
  How would your answer change if two points were charged on the 90\%
  loan?
\item
  Would your answer to (2) change if the borrower planned to own the
  property for only five years?
\end{enumerate}

\end{frame}

\begin{frame}[fragile]{Problem 1 answer}

\footnotesize

\begin{Shaded}
\begin{Highlighting}[]
\CommentTok{# 1. (assume loan balance = 1,000)}
\NormalTok{p1 =}\StringTok{ }\KeywordTok{PMT}\NormalTok{(}\FloatTok{0.08}\NormalTok{/}\DecValTok{12}\NormalTok{, }\DecValTok{12}\NormalTok{*}\DecValTok{25}\NormalTok{, -}\DecValTok{800}\NormalTok{, }\DecValTok{0}\NormalTok{) =}\StringTok{ }\FloatTok{6.17} \NormalTok{monthly}
\NormalTok{p2 =}\StringTok{ }\KeywordTok{PMT}\NormalTok{(}\FloatTok{0.085}\NormalTok{/}\DecValTok{12}\NormalTok{, }\DecValTok{12}\NormalTok{*}\DecValTok{25}\NormalTok{, -}\DecValTok{900}\NormalTok{) =}\StringTok{ }\FloatTok{7.25} \NormalTok{monthly}

\NormalTok{Difference in PV:}\StringTok{ }\NormalTok{PV1-PV2 =}\StringTok{ }\DecValTok{100}
\NormalTok{Difference in payment:}\StringTok{ }\NormalTok{p1-p2 =}\StringTok{ }\NormalTok{-}\FloatTok{1.07}

\NormalTok{Solve :}\StringTok{ }\ErrorTok{=}\DecValTok{12}\NormalTok{*}\KeywordTok{RATE}\NormalTok{(}\DecValTok{25}\NormalTok{*}\DecValTok{12}\NormalTok{, -}\FloatTok{1.07}\NormalTok{, }\DecValTok{100}\NormalTok{, }\DecValTok{0}\NormalTok{) =}\StringTok{ }\FloatTok{12.26}\NormalTok{\textbackslash{}%}

\CommentTok{# 2 Points:}
\NormalTok{p1 =}\StringTok{ }\KeywordTok{PMT}\NormalTok{(}\FloatTok{0.08}\NormalTok{/}\DecValTok{12}\NormalTok{, }\DecValTok{12}\NormalTok{*}\DecValTok{25}\NormalTok{, -}\DecValTok{800}\NormalTok{, }\DecValTok{0}\NormalTok{) =}\StringTok{ }\FloatTok{6.17} \NormalTok{monthly}
\NormalTok{p2 =}\StringTok{ }\KeywordTok{PMT}\NormalTok{(}\FloatTok{0.085}\NormalTok{/}\DecValTok{12}\NormalTok{, }\DecValTok{12}\NormalTok{*}\DecValTok{25}\NormalTok{, -}\DecValTok{900}\NormalTok{) =}\StringTok{ }\FloatTok{7.25} \NormalTok{monthly}

\NormalTok{Difference in PV:}\StringTok{ }\NormalTok{PV1-PV2 =}\StringTok{ }\DecValTok{100} \NormalTok{-}\StringTok{ }\FloatTok{0.02}\NormalTok{*}\DecValTok{900}
    \NormalTok{=}\StringTok{ }\DecValTok{82}
\NormalTok{Difference in payment:}\StringTok{ }\NormalTok{p1-p2 =}\StringTok{ }\NormalTok{-}\FloatTok{1.07}

\NormalTok{Solve :}\StringTok{ }\ErrorTok{=}\DecValTok{12}\NormalTok{*}\KeywordTok{RATE}\NormalTok{(}\DecValTok{25}\NormalTok{*}\DecValTok{12}\NormalTok{, -}\FloatTok{1.07}\NormalTok{, }\DecValTok{82}\NormalTok{, }\DecValTok{0}\NormalTok{) =}\StringTok{ }\FloatTok{15.35}\NormalTok{\textbackslash{}%}
\end{Highlighting}
\end{Shaded}

\end{frame}

\begin{frame}[fragile]{Problem 1 answer (continued)}

\begin{enumerate}
\def\labelenumi{\arabic{enumi}.}
\setcounter{enumi}{2}
\tightlist
\item
  Early repayment, loan balance after five years:
\end{enumerate}

\begin{Shaded}
\begin{Highlighting}[]
\NormalTok{p1 =}\StringTok{ }\KeywordTok{PMT}\NormalTok{(}\FloatTok{0.08}\NormalTok{/}\DecValTok{12}\NormalTok{, }\DecValTok{12}\NormalTok{*}\DecValTok{25}\NormalTok{, -}\DecValTok{800}\NormalTok{) =}\StringTok{ }\FloatTok{6.17} \NormalTok{monthly}
\NormalTok{p2 =}\StringTok{ }\KeywordTok{PMT}\NormalTok{(}\FloatTok{0.085}\NormalTok{/}\DecValTok{12}\NormalTok{, }\DecValTok{12}\NormalTok{*}\DecValTok{25}\NormalTok{, -}\DecValTok{900}\NormalTok{) =}\StringTok{ }\FloatTok{7.25} \NormalTok{monthly}

\NormalTok{FV1 =}\StringTok{ }\KeywordTok{FV}\NormalTok{(}\FloatTok{0.08}\NormalTok{/}\DecValTok{12}\NormalTok{, }\DecValTok{12}\NormalTok{*}\DecValTok{5}\NormalTok{, -}\FloatTok{6.17}\NormalTok{, }\DecValTok{800}\NormalTok{) =}\StringTok{ }\FloatTok{738.19}
\NormalTok{FV2 =}\StringTok{ }\KeywordTok{FV}\NormalTok{(}\FloatTok{0.085}\NormalTok{/}\DecValTok{12}\NormalTok{, }\DecValTok{12}\NormalTok{*}\DecValTok{5}\NormalTok{, -}\FloatTok{7.25}\NormalTok{, }\DecValTok{900}\NormalTok{) =}\StringTok{ }\FloatTok{835.08}

\NormalTok{Difference in PV:}\StringTok{ }\NormalTok{PV1-PV2 =}\StringTok{ }\DecValTok{100}
\NormalTok{Difference in payment:}\StringTok{ }\NormalTok{p1-p2 =}\StringTok{ }\NormalTok{-}\FloatTok{1.07}
\NormalTok{Difference in 5Y FV:}\StringTok{ }\NormalTok{FV1-FV2 =}\StringTok{ }\NormalTok{-}\FloatTok{96.89}

\NormalTok{Solve :}\StringTok{ }\ErrorTok{=}\DecValTok{12}\NormalTok{*}\KeywordTok{RATE}\NormalTok{(}\DecValTok{5}\NormalTok{*}\DecValTok{12}\NormalTok{, -}\FloatTok{1.07}\NormalTok{, }\DecValTok{100}\NormalTok{, -}\FloatTok{96.89}\NormalTok{) =}\StringTok{ }\FloatTok{12.42}\NormalTok{\textbackslash{}%}
\end{Highlighting}
\end{Shaded}

\normalsize

\end{frame}

\begin{frame}{Problem 3 (P.177)}

An investor obtained a fully amortizing mortgage 5 years ago for
\$95,000 at 11\% for 30 years. Mortgage rates have dropped, so that a
fully amortizing 25-year loan can be obtained at 10\%. There is no
prepayment penalty on the mortgage balance of the original loan, but
three points will be charged on the new loan and other closing costs
will be \$2,000. All payments are monthly.

\begin{enumerate}
\def\labelenumi{\arabic{enumi}.}
\tightlist
\item
  Should the borrower refinance if he plans to own the property for the
  remaining loan term? Assume that the investor borrows only an amount
  equal to the outstanding balance of the loan.
\item
  Would your answer to (1) change if he planned to own the property for
  only 5 more years?
\end{enumerate}

\end{frame}

\begin{frame}[fragile]{Problem 3 answer}

\footnotesize

\begin{Shaded}
\begin{Highlighting}[]
\CommentTok{# 1}
\NormalTok{=}\KeywordTok{PMT}\NormalTok{(}\FloatTok{0.11}\NormalTok{/}\DecValTok{12}\NormalTok{, }\DecValTok{30}\NormalTok{*}\DecValTok{12}\NormalTok{, }\DecValTok{95000}\NormalTok{, }\DecValTok{0}\NormalTok{) =}\StringTok{ }\NormalTok{-}\FloatTok{904.71}
\NormalTok{=}\KeywordTok{FV}\NormalTok{(}\FloatTok{0.11}\NormalTok{/}\DecValTok{12}\NormalTok{, }\DecValTok{5}\NormalTok{*}\DecValTok{12}\NormalTok{, -}\FloatTok{904.71}\NormalTok{, }\DecValTok{95000}\NormalTok{) =}\StringTok{ }\NormalTok{-}\DecValTok{92}\NormalTok{,}\FloatTok{306.19}
\NormalTok{=}\KeywordTok{PMT}\NormalTok{(}\FloatTok{0.10}\NormalTok{/}\DecValTok{12}\NormalTok{, }\DecValTok{25}\NormalTok{*}\DecValTok{12}\NormalTok{, }\FloatTok{92306.19}\NormalTok{, }\DecValTok{0}\NormalTok{) =}\StringTok{  }\NormalTok{-}\FloatTok{838.79}
\NormalTok{=}\DecValTok{12}\NormalTok{*}\KeywordTok{IRR}\NormalTok{(}\FloatTok{92306.19} \NormalTok{-}\StringTok{ }\FloatTok{92306.19}\NormalTok{*}\FloatTok{0.03} \NormalTok{-}\StringTok{ }\DecValTok{2000}\NormalTok{, -}\FloatTok{838.79}\NormalTok{, ...,}
    \NormalTok{(}\DecValTok{25}\NormalTok{*}\DecValTok{12} \NormalTok{payments) ...,-}\FloatTok{838.79}\NormalTok{, }\DecValTok{0}\NormalTok{) =}\StringTok{ }\FloatTok{10.69}\NormalTok{%}

\CommentTok{# 2}
\NormalTok{=}\KeywordTok{FV}\NormalTok{(}\FloatTok{0.11}\NormalTok{/}\DecValTok{12}\NormalTok{, }\DecValTok{10}\NormalTok{*}\DecValTok{12}\NormalTok{, -}\FloatTok{904.71}\NormalTok{, }\DecValTok{95000}\NormalTok{) =}\StringTok{ }\NormalTok{-}\DecValTok{87}\NormalTok{,}\FloatTok{648.83}
\NormalTok{=}\KeywordTok{PMT}\NormalTok{(}\FloatTok{0.10}\NormalTok{/}\DecValTok{12}\NormalTok{, }\DecValTok{20}\NormalTok{*}\DecValTok{12}\NormalTok{, }\FloatTok{87648.83}\NormalTok{, }\DecValTok{0}\NormalTok{) =}\StringTok{ }\NormalTok{-}\FloatTok{796.47}
\NormalTok{=}\KeywordTok{FV}\NormalTok{(}\FloatTok{0.10}\NormalTok{/}\DecValTok{12}\NormalTok{, }\DecValTok{5}\NormalTok{*}\DecValTok{12}\NormalTok{, -}\FloatTok{796.47}\NormalTok{, }\FloatTok{87648.83}\NormalTok{) =}\StringTok{ }\NormalTok{-}\DecValTok{82}\NormalTok{,}\FloatTok{533.10}
\NormalTok{=}\DecValTok{12}\NormalTok{*}\KeywordTok{IRR}\NormalTok{(}\FloatTok{87648.83} \NormalTok{-}\StringTok{ }\FloatTok{87648.83}\NormalTok{*}\FloatTok{0.03} \NormalTok{-}\StringTok{ }\DecValTok{2000}\NormalTok{, -}\FloatTok{796.47}\NormalTok{, ...,}
    \NormalTok{(}\DecValTok{5}\NormalTok{*}\DecValTok{12} \NormalTok{payments) ...,-}\FloatTok{796.47}\NormalTok{, -}\DecValTok{82}\NormalTok{,}\FloatTok{533.10}\NormalTok{) =}\StringTok{ }\FloatTok{11.29}\NormalTok{%}
\end{Highlighting}
\end{Shaded}

\end{frame}

\begin{frame}{Problem 4 (P.177)}

\small
Secondary Mortgage Purchasing Company (SMPC) wants to buy your mortgage
from the local savings and loan. The original balance of your mortgage
was \$140,000 and was obtained 5 years ago with monthly payments at 10\%
interest. The loan was to be fully amortized over 30 years.

\begin{enumerate}
\def\labelenumi{\arabic{enumi}.}
\tightlist
\item
  What should SMPC pay if it wants an 11\% return?
\item
  How would your answer to (1) change if SMPC expected the loan to be
  repaid after five years?
\end{enumerate}

\end{frame}

\begin{frame}[fragile]{Problem 4 answer}

\footnotesize

\begin{Shaded}
\begin{Highlighting}[]
\CommentTok{# 1}
\NormalTok{=}\KeywordTok{PMT}\NormalTok{(}\FloatTok{0.10}\NormalTok{/}\DecValTok{12}\NormalTok{, }\DecValTok{30}\NormalTok{*}\DecValTok{12}\NormalTok{, }\DecValTok{140000}\NormalTok{, }\DecValTok{0}\NormalTok{) =}\StringTok{ }\NormalTok{-}\FloatTok{1228.60}

\NormalTok{=}\KeywordTok{FV}\NormalTok{(}\FloatTok{0.10}\NormalTok{/}\DecValTok{12}\NormalTok{, }\DecValTok{5}\NormalTok{*}\DecValTok{12}\NormalTok{, -}\FloatTok{1228.60}\NormalTok{, }\DecValTok{140000}\NormalTok{) =}\StringTok{ }\NormalTok{-}\DecValTok{135}\NormalTok{,}\FloatTok{204.05}
\CommentTok{# Loan balance after 5 years}

\NormalTok{=}\KeywordTok{PV}\NormalTok{(}\FloatTok{0.11}\NormalTok{/}\DecValTok{12}\NormalTok{, }\DecValTok{25}\NormalTok{*}\DecValTok{12}\NormalTok{, -}\FloatTok{1228.60}\NormalTok{, }\DecValTok{0}\NormalTok{) =}\StringTok{ }\DecValTok{125}\NormalTok{,}\FloatTok{352.9}
\CommentTok{# PV of loan, given 11%, same payments and remaining 25y.}

\DecValTok{135}\NormalTok{,}\FloatTok{204.05} \NormalTok{-}\StringTok{ }\DecValTok{125}\NormalTok{,}\FloatTok{352.9} \NormalTok{=}\StringTok{ }\DecValTok{9}\NormalTok{,}\FloatTok{851.15}

\CommentTok{# 2}
\CommentTok{# Repayment after 5y}

\NormalTok{=}\KeywordTok{FV}\NormalTok{(}\FloatTok{0.11}\NormalTok{/}\DecValTok{12}\NormalTok{, }\DecValTok{5}\NormalTok{*}\DecValTok{12}\NormalTok{, -}\FloatTok{1228.60}\NormalTok{, }\FloatTok{125352.9}\NormalTok{) =}\StringTok{ }\NormalTok{-}\DecValTok{119}\NormalTok{,}\FloatTok{028.69}
\CommentTok{# Loan balance}

\NormalTok{=}\KeywordTok{PV}\NormalTok{(}\FloatTok{0.11}\NormalTok{/}\DecValTok{12}\NormalTok{, }\DecValTok{5}\NormalTok{*}\DecValTok{12}\NormalTok{, -}\FloatTok{1228.60}\NormalTok{, -}\FloatTok{119028.69}\NormalTok{) =}\StringTok{ }\DecValTok{125}\NormalTok{,}\FloatTok{352.9}
\CommentTok{# PV of loan is same (given same payment and i)}
\end{Highlighting}
\end{Shaded}

\end{frame}

\end{document}
