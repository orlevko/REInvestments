\documentclass[ignorenonframetext,]{beamer}
\setbeamertemplate{caption}[numbered]
\setbeamertemplate{caption label separator}{: }
\setbeamercolor{caption name}{fg=normal text.fg}
\beamertemplatenavigationsymbolsempty
\usepackage{lmodern}
\usepackage{amssymb,amsmath}
\usepackage{ifxetex,ifluatex}
\usepackage{fixltx2e} % provides \textsubscript
\ifnum 0\ifxetex 1\fi\ifluatex 1\fi=0 % if pdftex
\usepackage[T1]{fontenc}
\usepackage[utf8]{inputenc}
\else % if luatex or xelatex
\ifxetex
\usepackage{mathspec}
\else
\usepackage{fontspec}
\fi
\defaultfontfeatures{Ligatures=TeX,Scale=MatchLowercase}
\fi
% use upquote if available, for straight quotes in verbatim environments
\IfFileExists{upquote.sty}{\usepackage{upquote}}{}
% use microtype if available
\IfFileExists{microtype.sty}{%
\usepackage{microtype}
\UseMicrotypeSet[protrusion]{basicmath} % disable protrusion for tt fonts
}{}
\newif\ifbibliography
\usepackage{color}
\usepackage{fancyvrb}
\newcommand{\VerbBar}{|}
\newcommand{\VERB}{\Verb[commandchars=\\\{\}]}
\DefineVerbatimEnvironment{Highlighting}{Verbatim}{commandchars=\\\{\}}
% Add ',fontsize=\small' for more characters per line
\usepackage{framed}
\definecolor{shadecolor}{RGB}{248,248,248}
\newenvironment{Shaded}{\begin{snugshade}}{\end{snugshade}}
\newcommand{\KeywordTok}[1]{\textcolor[rgb]{0.13,0.29,0.53}{\textbf{{#1}}}}
\newcommand{\DataTypeTok}[1]{\textcolor[rgb]{0.13,0.29,0.53}{{#1}}}
\newcommand{\DecValTok}[1]{\textcolor[rgb]{0.00,0.00,0.81}{{#1}}}
\newcommand{\BaseNTok}[1]{\textcolor[rgb]{0.00,0.00,0.81}{{#1}}}
\newcommand{\FloatTok}[1]{\textcolor[rgb]{0.00,0.00,0.81}{{#1}}}
\newcommand{\ConstantTok}[1]{\textcolor[rgb]{0.00,0.00,0.00}{{#1}}}
\newcommand{\CharTok}[1]{\textcolor[rgb]{0.31,0.60,0.02}{{#1}}}
\newcommand{\SpecialCharTok}[1]{\textcolor[rgb]{0.00,0.00,0.00}{{#1}}}
\newcommand{\StringTok}[1]{\textcolor[rgb]{0.31,0.60,0.02}{{#1}}}
\newcommand{\VerbatimStringTok}[1]{\textcolor[rgb]{0.31,0.60,0.02}{{#1}}}
\newcommand{\SpecialStringTok}[1]{\textcolor[rgb]{0.31,0.60,0.02}{{#1}}}
\newcommand{\ImportTok}[1]{{#1}}
\newcommand{\CommentTok}[1]{\textcolor[rgb]{0.56,0.35,0.01}{\textit{{#1}}}}
\newcommand{\DocumentationTok}[1]{\textcolor[rgb]{0.56,0.35,0.01}{\textbf{\textit{{#1}}}}}
\newcommand{\AnnotationTok}[1]{\textcolor[rgb]{0.56,0.35,0.01}{\textbf{\textit{{#1}}}}}
\newcommand{\CommentVarTok}[1]{\textcolor[rgb]{0.56,0.35,0.01}{\textbf{\textit{{#1}}}}}
\newcommand{\OtherTok}[1]{\textcolor[rgb]{0.56,0.35,0.01}{{#1}}}
\newcommand{\FunctionTok}[1]{\textcolor[rgb]{0.00,0.00,0.00}{{#1}}}
\newcommand{\VariableTok}[1]{\textcolor[rgb]{0.00,0.00,0.00}{{#1}}}
\newcommand{\ControlFlowTok}[1]{\textcolor[rgb]{0.13,0.29,0.53}{\textbf{{#1}}}}
\newcommand{\OperatorTok}[1]{\textcolor[rgb]{0.81,0.36,0.00}{\textbf{{#1}}}}
\newcommand{\BuiltInTok}[1]{{#1}}
\newcommand{\ExtensionTok}[1]{{#1}}
\newcommand{\PreprocessorTok}[1]{\textcolor[rgb]{0.56,0.35,0.01}{\textit{{#1}}}}
\newcommand{\AttributeTok}[1]{\textcolor[rgb]{0.77,0.63,0.00}{{#1}}}
\newcommand{\RegionMarkerTok}[1]{{#1}}
\newcommand{\InformationTok}[1]{\textcolor[rgb]{0.56,0.35,0.01}{\textbf{\textit{{#1}}}}}
\newcommand{\WarningTok}[1]{\textcolor[rgb]{0.56,0.35,0.01}{\textbf{\textit{{#1}}}}}
\newcommand{\AlertTok}[1]{\textcolor[rgb]{0.94,0.16,0.16}{{#1}}}
\newcommand{\ErrorTok}[1]{\textcolor[rgb]{0.64,0.00,0.00}{\textbf{{#1}}}}
\newcommand{\NormalTok}[1]{{#1}}
\usepackage{longtable,booktabs}
\usepackage{caption}
% These lines are needed to make table captions work with longtable:
\makeatletter
\def\fnum@table{\tablename~\thetable}
\makeatother

% Prevent slide breaks in the middle of a paragraph:
\widowpenalties 1 10000
\raggedbottom

\AtBeginPart{
\let\insertpartnumber\relax
\let\partname\relax
\frame{\partpage}
}
\AtBeginSection{
\ifbibliography
\else
\let\insertsectionnumber\relax
\let\sectionname\relax
\frame{\sectionpage}
\fi
}
\AtBeginSubsection{
\let\insertsubsectionnumber\relax
\let\subsectionname\relax
\frame{\subsectionpage}
}

\setlength{\parindent}{0pt}
\setlength{\parskip}{6pt plus 2pt minus 1pt}
\setlength{\emergencystretch}{3em}  % prevent overfull lines
\providecommand{\tightlist}{%
\setlength{\itemsep}{0pt}\setlength{\parskip}{0pt}}
\setcounter{secnumdepth}{0}

\title{Mortgages}
\subtitle{Real-Estate Investments - Tutorial 2}
\author{Or Levkovich}
\date{15 September 2017}

\begin{document}
\frame{\titlepage}

\begin{frame}{Mortgages}

Mortgages are loans designated for investments in real estate.

Mortgage pricing (or interest, \(i\)) is derived from:

\begin{itemize}
\tightlist
\item
  Real rate of interest (\(r\)).
\item
  Inflation (\(f\))
\item
  Risk premium (\(p\)):

  \begin{itemize}
  \tightlist
  \item
    Possible future changes in interest rate or inflation rate.
  \item
    Prepayment risks.
  \item
    Liquidity risks
  \item
    Risk of default.
  \end{itemize}
\end{itemize}

Mortgage pricing follows : \(i = r + f + p\)

\end{frame}

\begin{frame}{Mortgages - important terms}

\begin{itemize}
\tightlist
\item
  Loan amount
\item
  Maturity date
\item
  Interest rates
\item
  Periodical payment = intrerest payment + amortization payment
\end{itemize}

\end{frame}

\begin{frame}{Fixed rate mortgage (FRM) mortgage loans}

\footnotesize

\begin{longtable}[]{@{}lll@{}}
\toprule
Type & Pay rate & Loan balance at maturity\tabularnewline
\midrule
\endhead
Fully amortizing & Greater than accrual rate & Fully
repaid\tabularnewline
Partially amortizing & Greater than accrual rate & Not fully
repaid\tabularnewline
Interest only & Equal to accrual rate & Equal to amount
borrowed\tabularnewline
Negative amortizing & Less than accrual rate & Greater than amount
borrowed\tabularnewline
\bottomrule
\end{longtable}

\textbf{Amortization}: The process of loan repayment over time.

\normalsize

\end{frame}

\begin{frame}{Types of FRM}

\textbf{Fully or Partially amortizing:} - \emph{Constant payments} - A
constant payment is calculated on an original loan amount, at a fixed
rate of interest for a given term (\emph{= Annuity mortgage}).

\textbf{Interest only:} - \emph{Zero Amortizing} - Constant payments
will be ``interest only''.

\textbf{Negative Amortizing:} - Constant payments are lower than the
periodic interest. Outstanding value at maturity is greater than
borrowed.

\textbf{Other:} - \emph{Constant amortization} - Payment is calculated
based on a fixed amortization rate, interest is calculated based on the
remaining outstanding amount (\emph{= Linear mortgage}).

\end{frame}

\section{Exercises (from the book)}\label{exercises-from-the-book}

\begin{frame}{Problem 1 (P.108)}

A borrower obtains a fully amortizing CPM loan for \$125,000 at 11
percent interest for 10 years. What will be the monthly payment on the
loan? If this loan had a maturity of 30 years, what would be the monthly
payment?

\end{frame}

\begin{frame}[fragile]{Problem 1 answer}

\begin{Shaded}
\begin{Highlighting}[]
\CommentTok{# 1}
\KeywordTok{pmt}\NormalTok{(}\FloatTok{0.11}\NormalTok{/}\DecValTok{12}\NormalTok{, }\DecValTok{10}\NormalTok{*}\DecValTok{12}\NormalTok{, -}\DecValTok{125000}\NormalTok{, }\DecValTok{0}\NormalTok{) =}\StringTok{ }
\DecValTok{1}\NormalTok{,}\FloatTok{721.88}

\KeywordTok{pmt}\NormalTok{(}\FloatTok{0.11}\NormalTok{/}\DecValTok{12}\NormalTok{, }\DecValTok{30}\NormalTok{*}\DecValTok{12}\NormalTok{, -}\DecValTok{125000}\NormalTok{, }\DecValTok{0}\NormalTok{) =}\StringTok{ }
\DecValTok{1}\NormalTok{,}\FloatTok{190.4}
\end{Highlighting}
\end{Shaded}

\end{frame}

\begin{frame}{Problem 6 (P.108)}

A 30-year fully amortizing mortgage loan was made 10 years ago for
\$75,000 at 6 percent interest. The borrower would like to prepay the
mortgage balance by \$10,000.

\begin{enumerate}
\def\labelenumi{\alph{enumi}.}
\tightlist
\item
  Assuming he can reduce his monthly mortgage payments, what is the new
  mortgage payment?
\item
  Assuming the loan maturity is shortened and using the original monthly
  payments, what is the new loan maturity?
\end{enumerate}

\end{frame}

\begin{frame}[fragile]{Problem 6 answer}

\small

\begin{Shaded}
\begin{Highlighting}[]
\CommentTok{# 6a }
\NormalTok{First, calculate payments value:}\StringTok{ }
\KeywordTok{pmt}\NormalTok{(}\FloatTok{0.06}\NormalTok{, }\DecValTok{30}\NormalTok{, -}\DecValTok{75000}\NormalTok{, }\DecValTok{0}\NormalTok{) =}\StringTok{ }\DecValTok{5}\NormalTok{,}\FloatTok{448.6}

\NormalTok{Then calculate PV of mortgage for the next }\DecValTok{20} \NormalTok{years:}
\KeywordTok{pv}\NormalTok{(}\FloatTok{0.06}\NormalTok{, }\DecValTok{20}\NormalTok{, -}\FloatTok{5448.6}\NormalTok{) =}\StringTok{ }\DecValTok{62}\NormalTok{,}\FloatTok{495.01}

\NormalTok{Then calculate new mortgage payment, after }\DecValTok{10}\NormalTok{,}\DecValTok{000} \NormalTok{prepayment:}
\KeywordTok{pmt}\NormalTok{(}\FloatTok{0.06}\NormalTok{, }\DecValTok{20}\NormalTok{, -(}\FloatTok{62495.01}\DecValTok{-10000}\NormalTok{), }\DecValTok{0}\NormalTok{) =}\StringTok{ }\DecValTok{4}\NormalTok{,}\FloatTok{576.75}


\CommentTok{# 6b}
\NormalTok{Calculate maturity given prepayment and original pmt:}
\KeywordTok{nper}\NormalTok{(}\FloatTok{0.06}\NormalTok{, -}\FloatTok{5448.6}\NormalTok{, (}\FloatTok{62495.01}\DecValTok{-10000}\NormalTok{)) =}\StringTok{ }\FloatTok{14.81}
\end{Highlighting}
\end{Shaded}

\normalsize

\end{frame}

\begin{frame}{Problem 8 (P.108)}

A fully amortizing mortgage is made for \$80,000 for 25 years. Total
monthly payments will be \$900 per month. What is the interest rate on
the loan?

\end{frame}

\begin{frame}[fragile]{Problem 8 answer}

\begin{Shaded}
\begin{Highlighting}[]
\CommentTok{# 8}
\DecValTok{12}\NormalTok{*}\KeywordTok{rate}\NormalTok{(}\DecValTok{25}\NormalTok{*}\DecValTok{12}\NormalTok{, -}\DecValTok{900}\NormalTok{, }\DecValTok{80000}\NormalTok{) =}\StringTok{ }\DecValTok{13}\NormalTok{%}
\end{Highlighting}
\end{Shaded}

\end{frame}

\section{Exercises (in Excel)}\label{exercises-in-excel}

\begin{frame}{Ex. 1}

\small

You are renting an apartment in the center of Amsterdam for a monthly
payment of 1000 Euro. You find out that your landlord wants to sell the
apartment, and you consider whether to buy it yourself.

The asking price for the property is 250,000 Euro. The bank agrees to
approve a \textbf{constant payment loan} for 100\% of the value, with an
interest rate of 4\% per year, compunded monthly, maturity in 20 years.

\begin{enumerate}
\def\labelenumi{\arabic{enumi}.}
\tightlist
\item
  Use the \emph{PMT} function in excel to recover what would be your
  monthly mortgage payment if you decide to buy.
\item
  You prefer to invest a monthly payment which is not higher than the
  value you currently pay for rent. Given the bank's offer, what would
  be the highest value you can pay for a house? (use Excel's \emph{PV}
  function).
\item
  The bank advices you on different types of mortgages, in which your
  monthly payment can be lower. For instance, \emph{Interest only}
  mortgage. What would be the monthly payment if you accept this loan?
\end{enumerate}

\normalsize

\end{frame}

\begin{frame}[fragile]{Ex. 1. Answers}

\begin{Shaded}
\begin{Highlighting}[]
\CommentTok{# 1 }
\KeywordTok{pmt}\NormalTok{(}\FloatTok{0.04}\NormalTok{/}\DecValTok{12}\NormalTok{, }\DecValTok{20}\NormalTok{*}\DecValTok{12}\NormalTok{, -}\DecValTok{250000}\NormalTok{, }\DecValTok{0}\NormalTok{)}
\NormalTok{=}\StringTok{ }\DecValTok{1}\NormalTok{,}\FloatTok{514.951}

\CommentTok{# 2}
\KeywordTok{pv}\NormalTok{(}\FloatTok{0.04}\NormalTok{/}\DecValTok{12}\NormalTok{, }\DecValTok{20}\NormalTok{*}\DecValTok{12}\NormalTok{, }\DecValTok{1000}\NormalTok{)}
\NormalTok{=}\StringTok{ }\DecValTok{165}\NormalTok{,}\FloatTok{021.9}

\CommentTok{# 3}
\KeywordTok{pmt}\NormalTok{(}\FloatTok{0.04}\NormalTok{/}\DecValTok{12}\NormalTok{, }\DecValTok{20}\NormalTok{*}\DecValTok{12}\NormalTok{, -}\DecValTok{250000}\NormalTok{, }\DecValTok{250000}\NormalTok{)}
\NormalTok{=}\StringTok{ }\FloatTok{833.3333}
\end{Highlighting}
\end{Shaded}

\end{frame}

\begin{frame}{Ex. 2}

\begin{enumerate}
\def\labelenumi{\arabic{enumi}.}
\item
  The government discusses limiting the LTV rate to 80\%. If this is the
  case, how long will you need to save before you can buy? (assuming
  that you save 1000 euro/month, same as your rate payments)
\item
  Alternatively, the bank offers you a \textbf{linear mortgage}. Open an
  excel file and calculate - What would be your monthly payments?
  Remember:
\end{enumerate}

\begin{itemize}
\tightlist
\item
  Monthly payment \emph{is not} constant. It is equal to \emph{PMT =
  Amortization (Fixed) + interest (changes with outstanding amount)}
\item
  conditions are as before: loan value = 250,000, T = 20*12, interest =
  4\%.
\item
  For more information please follow page 103 in the book.
\end{itemize}

\end{frame}

\begin{frame}[fragile]{Ex. 2. Answers}

\footnotesize

\begin{Shaded}
\begin{Highlighting}[]
\CommentTok{# 1}
\FloatTok{0.2}\NormalTok{*}\DecValTok{250000} \NormalTok{=}\StringTok{ }\DecValTok{50}\NormalTok{,}\DecValTok{000}
\KeywordTok{nper}\NormalTok{(}\FloatTok{0.04}\NormalTok{/}\DecValTok{12}\NormalTok{,+}\DecValTok{1000}\NormalTok{, }\DecValTok{50000}\NormalTok{)/}\DecValTok{12} \NormalTok{=}\StringTok{ }

\FloatTok{3.86} \NormalTok{years }
\NormalTok{(Assuming no price increase will occur in the meanwhile)}
\end{Highlighting}
\end{Shaded}

\normalsize

\begin{enumerate}
\def\labelenumi{\arabic{enumi}.}
\setcounter{enumi}{1}
\tightlist
\item
  See excel sheet.
\end{enumerate}

\end{frame}

\begin{frame}{Ex .3}

The bank eventually offers you a \textbf{constant payment} loan for
250,000, with a monthly payment of 1000 Euro and maturity in 40 years.

\begin{enumerate}
\def\labelenumi{\arabic{enumi}.}
\tightlist
\item
  What interest rate does this loan represent?
\end{enumerate}

It is 2008, interest rates are high and housing prices have been
increasing for almost 15 years. You believe it is a good time to invest
so you decide to buy. Unfortunately, in the following year the market
value of the house has dropped by 15\% (= 212,500).

What would be the monthly mortgage payment if you would have bought in
2009 instead of 2008?

\begin{enumerate}
\def\labelenumi{\arabic{enumi}.}
\setcounter{enumi}{1}
\tightlist
\item
  Under \emph{constant payment} loan?
\item
  Under \emph{Interest only} loan?
\end{enumerate}

\end{frame}

\begin{frame}[fragile]{Ex. 3. Answers}

\footnotesize

\begin{Shaded}
\begin{Highlighting}[]
\CommentTok{# 1. About 3.7%}
\DecValTok{12}\NormalTok{*}\KeywordTok{RATE}\NormalTok{(}\DecValTok{480}\NormalTok{, -}\DecValTok{1000}\NormalTok{, }\DecValTok{250000}\NormalTok{) =}\StringTok{ }
\FloatTok{3.71}\NormalTok{%}

\CommentTok{# 2.}
\KeywordTok{pmt}\NormalTok{(}\FloatTok{0.04}\NormalTok{/}\DecValTok{12}\NormalTok{, }\DecValTok{20}\NormalTok{*}\DecValTok{12}\NormalTok{, -}\DecValTok{212500}\NormalTok{, }\DecValTok{0}\NormalTok{) }
\NormalTok{=}\StringTok{ }\DecValTok{1}\NormalTok{,}\FloatTok{287.708}

\KeywordTok{pmt}\NormalTok{(}\FloatTok{0.04}\NormalTok{/}\DecValTok{12}\NormalTok{, }\DecValTok{20}\NormalTok{*}\DecValTok{12}\NormalTok{, -}\DecValTok{212500}\NormalTok{, }\DecValTok{0}\NormalTok{)-}
\KeywordTok{pmt}\NormalTok{(}\FloatTok{0.04}\NormalTok{/}\DecValTok{12}\NormalTok{, }\DecValTok{20}\NormalTok{*}\DecValTok{12}\NormalTok{, -}\DecValTok{250000}\NormalTok{, }\DecValTok{0}\NormalTok{)}
\NormalTok{=}\StringTok{ }\NormalTok{-}\FloatTok{227.242}

\CommentTok{# 3.}
\KeywordTok{pmt}\NormalTok{(}\FloatTok{0.04}\NormalTok{/}\DecValTok{12}\NormalTok{, }\DecValTok{20}\NormalTok{*}\DecValTok{12}\NormalTok{, -}\DecValTok{212500}\NormalTok{, }\DecValTok{212500}\NormalTok{) }
\NormalTok{=}\StringTok{ }\FloatTok{708.3333}

\KeywordTok{pmt}\NormalTok{(}\FloatTok{0.04}\NormalTok{/}\DecValTok{12}\NormalTok{, }\DecValTok{20}\NormalTok{*}\DecValTok{12}\NormalTok{, -}\DecValTok{250000}\NormalTok{, }\DecValTok{250000}\NormalTok{)-}
\KeywordTok{pmt}\NormalTok{(}\FloatTok{0.04}\NormalTok{/}\DecValTok{12}\NormalTok{, }\DecValTok{20}\NormalTok{*}\DecValTok{12}\NormalTok{, -}\DecValTok{212500}\NormalTok{, }\DecValTok{212500}\NormalTok{)}
 
\NormalTok{=}\StringTok{ }\NormalTok{-}\FloatTok{125.0}
\end{Highlighting}
\end{Shaded}

\normalsize

\end{frame}

\end{document}
