\documentclass[ignorenonframetext,]{beamer}
\setbeamertemplate{caption}[numbered]
\setbeamertemplate{caption label separator}{: }
\setbeamercolor{caption name}{fg=normal text.fg}
\beamertemplatenavigationsymbolsempty
\usepackage{lmodern}
\usepackage{amssymb,amsmath}
\usepackage{ifxetex,ifluatex}
\usepackage{fixltx2e} % provides \textsubscript
\ifnum 0\ifxetex 1\fi\ifluatex 1\fi=0 % if pdftex
\usepackage[T1]{fontenc}
\usepackage[utf8]{inputenc}
\else % if luatex or xelatex
\ifxetex
\usepackage{mathspec}
\else
\usepackage{fontspec}
\fi
\defaultfontfeatures{Ligatures=TeX,Scale=MatchLowercase}
\fi
% use upquote if available, for straight quotes in verbatim environments
\IfFileExists{upquote.sty}{\usepackage{upquote}}{}
% use microtype if available
\IfFileExists{microtype.sty}{%
\usepackage{microtype}
\UseMicrotypeSet[protrusion]{basicmath} % disable protrusion for tt fonts
}{}
\newif\ifbibliography
\usepackage{color}
\usepackage{fancyvrb}
\newcommand{\VerbBar}{|}
\newcommand{\VERB}{\Verb[commandchars=\\\{\}]}
\DefineVerbatimEnvironment{Highlighting}{Verbatim}{commandchars=\\\{\}}
% Add ',fontsize=\small' for more characters per line
\usepackage{framed}
\definecolor{shadecolor}{RGB}{248,248,248}
\newenvironment{Shaded}{\begin{snugshade}}{\end{snugshade}}
\newcommand{\KeywordTok}[1]{\textcolor[rgb]{0.13,0.29,0.53}{\textbf{{#1}}}}
\newcommand{\DataTypeTok}[1]{\textcolor[rgb]{0.13,0.29,0.53}{{#1}}}
\newcommand{\DecValTok}[1]{\textcolor[rgb]{0.00,0.00,0.81}{{#1}}}
\newcommand{\BaseNTok}[1]{\textcolor[rgb]{0.00,0.00,0.81}{{#1}}}
\newcommand{\FloatTok}[1]{\textcolor[rgb]{0.00,0.00,0.81}{{#1}}}
\newcommand{\ConstantTok}[1]{\textcolor[rgb]{0.00,0.00,0.00}{{#1}}}
\newcommand{\CharTok}[1]{\textcolor[rgb]{0.31,0.60,0.02}{{#1}}}
\newcommand{\SpecialCharTok}[1]{\textcolor[rgb]{0.00,0.00,0.00}{{#1}}}
\newcommand{\StringTok}[1]{\textcolor[rgb]{0.31,0.60,0.02}{{#1}}}
\newcommand{\VerbatimStringTok}[1]{\textcolor[rgb]{0.31,0.60,0.02}{{#1}}}
\newcommand{\SpecialStringTok}[1]{\textcolor[rgb]{0.31,0.60,0.02}{{#1}}}
\newcommand{\ImportTok}[1]{{#1}}
\newcommand{\CommentTok}[1]{\textcolor[rgb]{0.56,0.35,0.01}{\textit{{#1}}}}
\newcommand{\DocumentationTok}[1]{\textcolor[rgb]{0.56,0.35,0.01}{\textbf{\textit{{#1}}}}}
\newcommand{\AnnotationTok}[1]{\textcolor[rgb]{0.56,0.35,0.01}{\textbf{\textit{{#1}}}}}
\newcommand{\CommentVarTok}[1]{\textcolor[rgb]{0.56,0.35,0.01}{\textbf{\textit{{#1}}}}}
\newcommand{\OtherTok}[1]{\textcolor[rgb]{0.56,0.35,0.01}{{#1}}}
\newcommand{\FunctionTok}[1]{\textcolor[rgb]{0.00,0.00,0.00}{{#1}}}
\newcommand{\VariableTok}[1]{\textcolor[rgb]{0.00,0.00,0.00}{{#1}}}
\newcommand{\ControlFlowTok}[1]{\textcolor[rgb]{0.13,0.29,0.53}{\textbf{{#1}}}}
\newcommand{\OperatorTok}[1]{\textcolor[rgb]{0.81,0.36,0.00}{\textbf{{#1}}}}
\newcommand{\BuiltInTok}[1]{{#1}}
\newcommand{\ExtensionTok}[1]{{#1}}
\newcommand{\PreprocessorTok}[1]{\textcolor[rgb]{0.56,0.35,0.01}{\textit{{#1}}}}
\newcommand{\AttributeTok}[1]{\textcolor[rgb]{0.77,0.63,0.00}{{#1}}}
\newcommand{\RegionMarkerTok}[1]{{#1}}
\newcommand{\InformationTok}[1]{\textcolor[rgb]{0.56,0.35,0.01}{\textbf{\textit{{#1}}}}}
\newcommand{\WarningTok}[1]{\textcolor[rgb]{0.56,0.35,0.01}{\textbf{\textit{{#1}}}}}
\newcommand{\AlertTok}[1]{\textcolor[rgb]{0.94,0.16,0.16}{{#1}}}
\newcommand{\ErrorTok}[1]{\textcolor[rgb]{0.64,0.00,0.00}{\textbf{{#1}}}}
\newcommand{\NormalTok}[1]{{#1}}
\usepackage{longtable,booktabs}
\usepackage{caption}
% These lines are needed to make table captions work with longtable:
\makeatletter
\def\fnum@table{\tablename~\thetable}
\makeatother

% Prevent slide breaks in the middle of a paragraph:
\widowpenalties 1 10000
\raggedbottom

\AtBeginPart{
\let\insertpartnumber\relax
\let\partname\relax
\frame{\partpage}
}
\AtBeginSection{
\ifbibliography
\else
\let\insertsectionnumber\relax
\let\sectionname\relax
\frame{\sectionpage}
\fi
}
\AtBeginSubsection{
\let\insertsubsectionnumber\relax
\let\subsectionname\relax
\frame{\subsectionpage}
}

\setlength{\parindent}{0pt}
\setlength{\parskip}{6pt plus 2pt minus 1pt}
\setlength{\emergencystretch}{3em}  % prevent overfull lines
\providecommand{\tightlist}{%
\setlength{\itemsep}{0pt}\setlength{\parskip}{0pt}}
\setcounter{secnumdepth}{0}

\title{Mortgages}
\subtitle{Real-Estate Investments - Tutorial 2}
\author{Or Levkovich}
\date{15 September 2017}

\begin{document}
\frame{\titlepage}

\begin{frame}{Mortgages}

Mortgages are loans designated for investments in real estate.

Mortgage pricing (or interest, \(i\)) is derived from:

\begin{itemize}
\tightlist
\item
  Real rate of interest (\(r\)).
\item
  Inflation (\(f\))
\item
  Risk premium (\(p\)):

  \begin{itemize}
  \tightlist
  \item
    Possible future changes in interest rate or inflation rate.
  \item
    Prepayment risks.
  \item
    Liquidity risks
  \item
    Risk of default.
  \end{itemize}
\end{itemize}

Mortgage pricing follows : \(i = r + f + p\)

\end{frame}

\begin{frame}{Mortgages - important terms}

\begin{itemize}
\tightlist
\item
  Loan amount
\item
  Maturity date
\item
  Interest rates
\item
  Periodical payment = intrerest payment + amortization payment
\end{itemize}

\end{frame}

\begin{frame}{Fixed rate mortgage (FRM) mortgage loans}

\footnotesize

\begin{longtable}[]{@{}lll@{}}
\toprule
Type & Pay rate & Loan balance at maturity\tabularnewline
\midrule
\endhead
Fully amortizing & Greater than accrual rate & Fully
repaid\tabularnewline
Partially amortizing & Greater than accrual rate & Not fully
repaid\tabularnewline
Interest only & Equal to accrual rate & Equal to amount
borrowed\tabularnewline
Negative amortizing & Less than accrual rate & Greater than amount
borrowed\tabularnewline
\bottomrule
\end{longtable}

\textbf{Amortization}: The process of loan repayment over time.

\normalsize

\end{frame}

\begin{frame}{Types of FRM}

\textbf{Fully or Partially amortizing:} - \emph{Constant payments} - A
constant payment is calculated on an original loan amount, at a fixed
rate of interest for a given term (\emph{= Annuity mortgage}).

\textbf{Interest only:} - \emph{Zero Amortizing} - Constant payments
will be ``interest only''.

\textbf{Negative Amortizing:} - Constant payments are lower than the
periodic interest. Outstanding value at maturity is greater than
borrowed.

\textbf{Other:} - \emph{Constant amortization} - Payment is calculated
based on a fixed amortization rate, interest is calculated based on the
remaining outstanding amount (\emph{= Linear mortgage}).

\end{frame}

\section{Exercises (in Excel)}\label{exercises-in-excel}

\begin{frame}{Ex. 1}

\small

You are renting an apartment in the center of Amsterdam for a monthly
payment of 1000 Euro. You find out that your landlord wants to sell the
apartment, and you consider whether to buy it yourself.

The asking price for the property is 250,000 Euro. The bank agrees to
approve a \textbf{constant payment loan} for 100\% of the value, with an
interest rate of 4\% per year, compunded monthly, maturity in 20 years.

\begin{enumerate}
\def\labelenumi{\arabic{enumi}.}
\tightlist
\item
  Use the \emph{PMT} function in excel to recover what would be your
  monthly mortgage payment if you decide to buy.
\item
  You prefer to invest a monthly payment which is not higher than the
  value you currently pay for rent. Given the bank's offer, what would
  be the highest value you can pay for a house? (use Excel's \emph{PV}
  function).
\item
  The bank advices you on different types of mortgages, in which your
  monthly payment can be lower. For instance, \emph{Interest only}
  mortgage. What would be the monthly payment if you accept this loan?
\end{enumerate}

\normalsize

\end{frame}

\begin{frame}[fragile]{Ex. 1. Answers}

\begin{Shaded}
\begin{Highlighting}[]
\CommentTok{# 1 }
\KeywordTok{pmt}\NormalTok{(}\FloatTok{0.04}\NormalTok{/}\DecValTok{12}\NormalTok{, }\DecValTok{20}\NormalTok{*}\DecValTok{12}\NormalTok{, -}\DecValTok{250000}\NormalTok{, }\DecValTok{0}\NormalTok{)}
\NormalTok{=}\StringTok{ }\DecValTok{1}\NormalTok{,}\FloatTok{514.951}

\CommentTok{# 2}
\KeywordTok{pv}\NormalTok{(}\FloatTok{0.04}\NormalTok{/}\DecValTok{12}\NormalTok{, }\DecValTok{20}\NormalTok{*}\DecValTok{12}\NormalTok{, }\DecValTok{1000}\NormalTok{)}
\NormalTok{=}\StringTok{ }\DecValTok{165}\NormalTok{,}\FloatTok{021.9}

\CommentTok{# 3}
\KeywordTok{pmt}\NormalTok{(}\FloatTok{0.04}\NormalTok{/}\DecValTok{12}\NormalTok{, }\DecValTok{20}\NormalTok{*}\DecValTok{12}\NormalTok{, -}\DecValTok{250000}\NormalTok{, }\DecValTok{250000}\NormalTok{)}
\NormalTok{=}\StringTok{ }\FloatTok{833.3333}
\end{Highlighting}
\end{Shaded}

\end{frame}

\begin{frame}{Ex. 2}

\begin{enumerate}
\def\labelenumi{\arabic{enumi}.}
\tightlist
\item
  Alternatively, the bank offers you a \textbf{linear mortgage}. Open an
  excel file and calculate - What would be your monthly payments?
  Remember:
\end{enumerate}

\begin{itemize}
\tightlist
\item
  Monthly payment \emph{is not} constant. It is equal to \emph{PMT =
  Amortization (Fixed) + interest (changes with outstanding amount)}
\item
  conditions are as before: loan value = 250,000, T = 20*12, interest =
  4\%.
\item
  For more information please follow page 103 in the book.
\end{itemize}

\end{frame}

\begin{frame}{Ex. 2. Answers}

\begin{enumerate}
\def\labelenumi{\arabic{enumi}.}
\tightlist
\item
  See excel sheet.
\end{enumerate}

\end{frame}

\begin{frame}{Ex .3}

It is 2007, interest rates are high and housing prices have been
increasing for almost 15 years. You believe it is a good time to invest,
and that interest rates will remain high in the future (perhaps even
increase!).

\begin{enumerate}
\def\labelenumi{\arabic{enumi}.}
\tightlist
\item
  Which of the fixed-rate mortgage loan types (\emph{constant payment,
  interest only, constant amortization}) would you prefer to accept,
  given your expectations? (and why?)
\end{enumerate}

\end{frame}

\begin{frame}{Ex. 3. Answers}

\begin{enumerate}
\def\labelenumi{\arabic{enumi}.}
\item
\end{enumerate}

\begin{itemize}
\tightlist
\item
  We assume interest rates to remain stable, or increase.
\item
  ``Constant payment'' and ``constant amortization'' loans lead to full
  amortization at maturity. ``Interest only'' loan leads to zero
  amortization, and the outstanding amount has to be paid in full.
\item
  Assuming that interest rates will increase implies that mortgage
  payments of \emph{all} loans will reflect gain (the difference between
  the loan rate and market rate), but ``interest only'' loan will
  reflect the highest gain.
\end{itemize}

\end{frame}

\end{document}
